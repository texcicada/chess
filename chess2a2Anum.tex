\documentclass[12pt]{article}
\usepackage{xcolor}
\pagecolor{red!3}
\usepackage{graphicx}
\usepackage{fontspec}
\setmainfont{Noto Serif}
\newfontfamily\fchess{GaramondExtended}%FreeSerif}%Meiryo}%Arial Unicode MS}%[Scale=1.1]%DejaVu Sans Mono}
\newfontface\cfont[Mapping=latin-to-chess,Colour=blue,Scale=2]{Noto Sans Symbols}



\usepackage{tikz}
\usetikzlibrary{positioning}
\usetikzlibrary{math}
\usetikzlibrary{shapes.geometric}



\newcommand\wsqc{red!3}
\newcommand\bsqc{blue!15}

\expandafter\newcommand\csname yc1\endcsname{\bsqc}
\expandafter\newcommand\csname yc2\endcsname{\wsqc}


%-----------------------------------------------------
% the board
% orientation of the board = bottom right is white square for each player
\newcommand\drawtheboard{%
\begin{tikzpicture}[node distance=0pt,remember picture]
%start = anchor: bottom left = black
\node[rectangle,fill={\csname yc1\endcsname},] (a1) {{\fchess \phantom ♚ }} ;
%file 1:
\foreach \xx/\x/\fc in {b/a/2,c/b/1,d/c/2,e/d/1,f/e/2,g/f/1,h/g/2}
{
\node[rectangle,fill={\csname yc\fc\endcsname},right=of \x 1] (\xx 1) {{\fchess \phantom ♚ }};
}
%aceg:
\foreach \rank in {a,c,e,g}
{
\foreach \yy/\y/\fc in {2/1/2,3/2/1,4/3/2,5/4/1,6/5/2,7/6/1,8/7/2}
{
\node[rectangle,fill={\csname yc\fc\endcsname},above=of \rank\y] (\rank\yy) {{\fchess \phantom ♚ }};
}
}%----
%bdfh:
\foreach \rank in {b,d,f,h}
{
\foreach \yy/\y/\fc in {2/1/1,3/2/2,4/3/1,5/4/2,6/5/1,7/6/2,8/7/1}
{
\node[rectangle,fill={\csname yc\fc\endcsname},above=of \rank\y] (\rank\yy) {{\fchess \phantom ♚ }};
}
}
% the letters
\foreach \letter in {a,b,c,d,e,f,g,h}
{
\node[rectangle,fill={\csname yc2\endcsname},below=of \letter 1] (\letter) {{\scriptsize\strut \letter}};
}
% the numbers
\foreach \number in {1,2,3,4,5,6,7,8}
{
\node[rectangle,fill={\csname yc2\endcsname},left=of a\number] (\number) {{\scriptsize\number}};
}
% the edge of the board
\draw  (a1.south west) rectangle (h8.north east);
\end{tikzpicture}
}


%-----------------------------------------------------
% fill the board with numbers
\newcount\squarenum
\newcommand\drawtheboardnum{%
% number the squares
\begin{tikzpicture}[node distance=0pt,remember picture, overlay]
\foreach \letter [count=\x] in {a,b,c,d,e,f,g,h}
{
\foreach \number [count=\y] in {1,2,3,4,5,6,7,8}
{
\pgfmathparse{(8*(\y-1) + \x)}
\pgfmathfloatparsenumber{\pgfmathresult}
\pgfmathfloattoint{\pgfmathresult}
\node at (\letter\number) {\pgfmathresult};%
}
}
\end{tikzpicture}
}







%============================ exposed
\newcommand\drawthepath[1]{%
\begin{tikzpicture}[remember picture,overlay]
% arrow indicator path
\foreach \p in {#1}
\node at (\p) [fill=red!50!yellow!90, rounded corners, opacity=0.6] {{\fchess \phantom ♚ }};
\end{tikzpicture}
}
%============================ attack
\newcommand\drawthepathb[1]{%
\begin{tikzpicture}[remember picture,overlay]
% arrow indicator path
\foreach \p [count=\x] in {#1}
\ifnum\x=1%
{\node at (\p) [draw, circle, red!30!yellow, very thick] {{\fchess \phantom ♚ }};}
\else
{\node at (\p) [fill=yellow!90!green, rounded corners, opacity=0.9] {{\fchess \phantom ♚ }};}
\fi;
\end{tikzpicture}
}

%============================ coverage
\newcommand\drawthepathc[1]{%
\begin{tikzpicture}[remember picture,overlay]
% arrow indicator path
\foreach \p [count=\x] in {#1}
\ifnum\x=1%
{\node at (\p) [draw, circle, red!3, thick] {{\fchess \phantom ♚ }};}
\else
{\node at (\p) [fill=violet!52] {{\fchess \phantom ♚ }};}
\fi;
\end{tikzpicture}
}



%-----------------------------------------------------
\newcommand\setwhitepieces[1]{%
\begin{tikzpicture}[remember picture,overlay]
%white:
\foreach \sq/\cp/\cpn in {#1}
{
\node (\cpn) at (\sq) [inner sep=-0.5pt] {{\fchess \cp}};
}
\end{tikzpicture}
}

%-----------------------------------------------------
\newcommand\setblackpieces[1]{%
\begin{tikzpicture}[remember picture,overlay]
%black:
\foreach \sq/\cp/\cpn in {#1}
{
\node (\cpn) at (\sq) [inner sep=-0.5pt] {{\fchess \cp}};
}
\end{tikzpicture}
}

%-----------------------------------------------------
\newcommand\drawthearrow[2]{%start piece, end piece
\begin{tikzpicture}[remember picture,overlay]
\draw[->,very	 thick, blue] (#1)--(#2);
\end{tikzpicture}
}
\newcommand\drawthearrowb[2]{%start piece, end piece
\begin{tikzpicture}[remember picture,overlay]
\draw[->,very	 thick, red] (#1)--(#2);
\end{tikzpicture}
}



%%the pieces:
%%♔♕♖♗♘♙♚♛♜♝♞♟
\newcommand\setupboard{%
%Queens are on their own colours: d1 and d8.
\setwhitepieces{a1/♖/wqr,b1/♘/wqn,c1/♗/wqb,d1/♕/wq,e1/♔/wk,f1/♗/wkb,g1/♘/wkn,h1/♖/wkr}
\setwhitepieces{a2/♙,b2/♙,c2/♙,d2/♙,e2/♙,f2/♙,g2/♙,h2/♙}
\setblackpieces{a8/♜/bqr,b8/♞/bqn,c8/♝/bqb,d8/♛/bq,e8/♚/bk,f8/♝/bkb,g8/♞/bkn,h8/♜/bkr}
\setblackpieces{a7/♟,b7/♟,c7/♟,d7/♟,e7/♟,f7/♟,g7/♟,h7/♟}
}

% nodetext(=glyph)/nodename
%\newcommand\wq{♕/wq}
%\newcommand\bk{♚/bk}

\begin{document}
\section*{{\cfont{cwn}} A file about chess}
Chessboards with TikZ.
\bigskip

\drawtheboard
\drawthepath{a4,b5,c6,d7,e8}
\setwhitepieces{a4/♕/wq}
\setblackpieces{e8/♚/bk}
\drawthearrow{wq}{bk}

\bigskip
\drawtheboard
\drawtheboardnum

\bigskip
\drawtheboard
\setupboard


\bigskip
\drawtheboard
\drawthepathb{a4,b5,c6,d7,e8}
\setwhitepieces{a1/♖/wqr,b1/♘/wqn,c1/♗/wqb,a4/♕/wq,e1/♔/wk,f1/♗/wkb,g1/♘/wkn,h1/♖/wkr}
\setwhitepieces{a2/♙,b2/♙,c3/♙,d2/♙,e2/♙,f2/♙,g2/♙,h2/♙}
\setblackpieces{a8/♜/bqr,b8/♞/bqn,c8/♝/bqb,d8/♛/bq,e8/♚/bk,f8/♝/bkb,g8/♞/bkn,h8/♜/bkr}
\setblackpieces{a7/♟,b7/♟,c7/♟,d6/♟,e7/♟,f7/♟,g7/♟,h7/♟}
\drawthearrowb{wq}{bk}



\begin{figure}
\begin{center}
\drawtheboard
\drawthepathc{d4,c4,b4,a4}
\drawthepathc{d4,e4,f4,g4,h4}
\drawthepathc{d4,d5,d6,d7,d8}
\drawthepathc{d4,d3,d2,d1}
%
\drawthepathc{d4,c5,b6,a7}
\drawthepathc{d4,e5,f6,g7,h8}
\drawthepathc{d4,e3,f2,g1}
\drawthepathc{d4,c3,b2,a1}
%
\setwhitepieces{d4/♕/wq}
\end{center}
\caption{Queen's coverage}
\end{figure}



%\begin{center}
\begin{tabular}{ll}
% King
\begin{tabular}{l}
\drawtheboard
\drawthepathc{d4,c3,d3,e3,e4,e5,d5,c5,c4}
%
\setwhitepieces{d4/♔/wk} \\
\end{tabular}
& \begin{minipage}{4cm}\textbf{King}'s coverage is the adjacent square.\end{minipage} \\
\end{tabular}
%\end{center}

% Queen
%\begin{center}
\begin{tabular}{ll}
\begin{tabular}{l}
\drawtheboard
\drawthepathc{d4,c4,b4,a4}
\drawthepathc{d4,e4,f4,g4,h4}
\drawthepathc{d4,d5,d6,d7,d8}
\drawthepathc{d4,d3,d2,d1}
%
\drawthepathc{d4,c5,b6,a7}
\drawthepathc{d4,e5,f6,g7,h8}
\drawthepathc{d4,e3,f2,g1}
\drawthepathc{d4,c3,b2,a1}
%
\setwhitepieces{d4/♕/wq} \\
\end{tabular}
& \begin{minipage}{4cm}\textbf{Queen}'s coverage is along the diagonals, and the horizontals and verticals.\end{minipage} \\
\end{tabular}
%\end{center}

% Bishop
%\begin{center}
\begin{tabular}{ll}
\begin{tabular}{l}
\drawtheboard
%\drawthepathc{d4,c4,b4,a4}
%\drawthepathc{d4,e4,f4,g4,h4}
%\drawthepathc{d4,d5,d6,d7,d8}
%\drawthepathc{d4,d3,d2,d1}
%
\drawthepathc{d4,c5,b6,a7}
\drawthepathc{d4,e5,f6,g7,h8}
\drawthepathc{d4,e3,f2,g1}
\drawthepathc{d4,c3,b2,a1}
%
\setwhitepieces{d4/♗/wqb} \\
\end{tabular}
& \begin{minipage}{4cm}\textbf{Bishop}'s coverage is along the diagonals of its colour.\end{minipage} \\
\end{tabular}
%\end{center}

%
% Knight
%\begin{center}
\begin{tabular}{ll}
\begin{tabular}{l}
\drawtheboard
\drawthepathc{d4,c6}
\drawthepathc{d4,e6}
\drawthepathc{d4,f5}
\drawthepathc{d4,f3}
%
\drawthepathc{d4,e2}
\drawthepathc{d4,c2}
\drawthepathc{d4,b3}
\drawthepathc{d4,b5}
%
\setwhitepieces{d4/♘/wqn} 
\begin{tikzpicture}[remember picture,overlay]
\draw[->,very	 thick, blue, bend left=45] (d4.west) to (b5.south);
\draw[->,very	 thick, blue, bend right=45] (d4.west) to (b3.north);
%
\draw[->,very	 thick, blue, bend right=45] (d4.north) to (c6.east);
\draw[->,very	 thick, blue, bend left=45] (d4.north) to (e6.west);
%
\draw[->,very	 thick, blue, bend right=45] (d4.east) to (f5.south);
\draw[->,very	 thick, blue, bend left=45] (d4.east) to (f3.north);
%
\draw[->,very	 thick, blue, bend left=45] (d4.south) to (c2.east);
\draw[->,very	 thick, blue, bend right=45] (d4.south) to (e2.west);
\end{tikzpicture}
%
\\
\end{tabular}
& 	\begin{minipage}{4cm}\textbf{Knight}'s coverage can leap over other pieces, and is L-shaped: two squares horizontally or vertically, then one square at right angles, either to the left or right.\end{minipage} \\
\end{tabular}
%\end{center}

%
% Rook
%\begin{center}
\begin{tabular}{ll}
\begin{tabular}{l}
\drawtheboard
\drawthepathc{d4,c4,b4,a4}
\drawthepathc{d4,e4,f4,g4,h4}
\drawthepathc{d4,d5,d6,d7,d8}
\drawthepathc{d4,d3,d2,d1}
%
%\drawthepathc{d4,c5,b6,a7}
%\drawthepathc{d4,e5,f6,g7,h8}
%\drawthepathc{d4,e3,f2,g1}
%\drawthepathc{d4,c3,b2,a1}
%
\setwhitepieces{d4/♖/wqr} \\
\end{tabular}
& \begin{minipage}{4cm}\textbf{Rook}'s coverage is the horizontals and verticals.\end{minipage} \\
%
\end{tabular}
%\end{center}

% Pawn 
%\begin{center}
\begin{tabular}{ll}
\begin{tabular}{l}
\drawtheboard
\drawthepathc{d4,d5}
%
\setwhitepieces{d4/♙/wqrb} 
\begin{tikzpicture}[remember picture,overlay]
\draw[->,very	 thick, blue] (wqrb.north west) to (c5.center);
\draw[->,very	 thick, blue] (wqrb.north east) to (e5.center);
\end{tikzpicture}
\\
\end{tabular}
& \begin{minipage}{4cm}A \textbf{pawn} can only move forward one square at a time if no other piece blocks the way (initial move may be two squares forward); attack may be diagonally forward.\end{minipage} \\
%
\end{tabular}
%\end{center}

\newpage

Font mappings (latin-to-chess):


\begin{tabular}{cll}
1 & {cwk}  $\to$  & \cfont{cwk} \\
2 & {cwq}  $\to$  & \cfont{cwq} \\
3 & {cwr}  $\to$  & \cfont{cwr} \\
4 & {cwb}  $\to$  & \cfont{cwb} \\
5 & {cwn}  $\to$  & \cfont{cwn} \\
6 & {cwp}  $\to$  & \cfont{cwp} \\
7 & {cbk}  $\to$  & \cfont{cbk} \\
8 & {cbq}  $\to$  & \cfont{cbq} \\
9 & {cbr}  $\to$  & \cfont{cbr} \\
10 & {cbb}  $\to$  & \cfont{cbb} \\
11 & {cbn}  $\to$  & \cfont{cbn} \\
12 & {cbp}  $\to$  & \cfont{cbp} \\
\end{tabular}

\begin{tabular}{cll}
13 & {sspd}  $\to$  & \cfont{sspd} \\
14 & {thrt}  $\to$  & \cfont{thrt} \\
15 & {tdmnd}  $\to$  & \cfont{tdmnd} \\
16 & {sclb}  $\to$  & \cfont{sclb} \\
17 & {tspd}  $\to$  & \cfont{tspd} \\
18 & {shrt}  $\to$  & \cfont{shrt} \\
19 & {sdmnd}  $\to$  & \cfont{sdmnd} \\
20 & {tclb}  $\to$  & \cfont{tclb} \\
\end{tabular}

\begin{tabular}{cll}
21 & {d1}  $\to$  & \cfont{d1} \\
22 & {d2}  $\to$  & \cfont{d2} \\
23 & {d3}  $\to$  & \cfont{d3} \\
24 & {d4}  $\to$  & \cfont{d4} \\
25 & {d5}  $\to$  & \cfont{d5} \\
26 & {d6}  $\to$  & \cfont{d6} \\
\end{tabular}

\begin{tabular}{cll}
27 & {llaw}  $\to$  & \cfont{llaw} \\
28 & {cchem}  $\to$  & \cfont{cchem} \\
29 & {bbot}  $\to$  & \cfont{bbot} \\
\end{tabular}

\begin{tabular}{cll}
30 & {ssocc}  $\to$  & \cfont{ssocc} \\
31 & {bbas}  $\to$  & \cfont{bbas} \\
32 & {sstar}  $\to$  & \cfont{sstar} \\
\end{tabular}


\end{document}
